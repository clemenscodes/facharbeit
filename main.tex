\documentclass[a4paper,12pt]{article}
\usepackage[utf8]{inputenc}
\usepackage[T1]{fontenc}
\usepackage[ngerman]{babel}
\usepackage{graphicx}
\usepackage{geometry}
\usepackage{setspace}
\usepackage{float}
\usepackage{hyperref}
\usepackage{tabularx}
\usepackage{booktabs}
\usepackage{cite}

\geometry{a4paper, left=2cm, right=2cm, top=2cm, bottom=2cm}
\onehalfspacing

\usepackage{helvet}
\renewcommand\familydefault{\sfdefault}

\newcommand{\autor}{Elyssa Jane Dean}
\newcommand{\geburtsdatum}{27.09.1992}
\newcommand{\arbeitgeber}{IUSCare GmbH Ambulante Intensivpflege}
\newcommand{\kursbezeichnung}{FAI: Fachkraft für außerklinische Intensivpflege}
\newcommand{\kurszeitraum}{27.01.2025 - 04.02.2025}
\newcommand{\abgabedatum}{11.03.2025}

\title{\textbf{Facharbeit}}
\author{
    \autor \\
    Geburtsdatum: \geburtsdatum \\
    Arbeitgeber: \arbeitgeber \\
    Kursbezeichnung: \kursbezeichnung \\
    Zeitraum des Kurses: \kurszeitraum \\
    Abgabedatum: \abgabedatum
}
\date{}

\begin{document}

\begin{titlepage}
	\centering
	\vspace*{2cm}
	\vspace{1cm}

	{\Huge \textbf{Facharbeit} \par}
	\vspace{1.5cm}

	{\Large \textbf{\autor} \par}
	\vspace{0.5cm}

	{\large Geburtsdatum: \geburtsdatum \par}
	\vspace{0.5cm}

	{\large Arbeitgeber: \arbeitgeber \par}
	\vspace{0.5cm}

	{\large Kursbezeichnung: \kursbezeichnung \par}
	\vspace{0.5cm}

	{\large Zeitraum des Kurses: \kurszeitraum \par}
	\vspace{0.5cm}

	{\large Abgabedatum: \abgabedatum \par}
	\vspace{2cm}

	\vfill
	{\large \today \par}
\end{titlepage}

\tableofcontents
\newpage

\section{Einleitung}
In dieser Facharbeit wird das Thema XYZ behandelt. Der Grund für die
Erstellung dieser Arbeit liegt in meinem persönlichen Interesse an
	[Thema einfügen] sowie der beruflichen Relevanz im Rahmen meiner
Tätigkeit bei der \arbeitgeber. Durch die Teilnahme am Kurs
\kursbezeichnung konnte ich mein Wissen in diesem Bereich vertiefen
und möchte die gewonnenen Erkenntnisse in dieser Arbeit darlegen.

Ich bin \autor, geboren am \geburtsdatum, und arbeite seit [Jahr]
bei der \arbeitgeber. Meine berufliche Laufbahn hat mich immer wieder
mit [Thema einfügen] in Berührung gebracht, was mich dazu motiviert
hat, mich intensiver mit dieser Thematik auseinanderzusetzen.
Diese Facharbeit ist daher sowohl eine Zusammenfassung meiner
bisherigen Erfahrungen als auch eine Reflexion über zukünftige
Entwicklungen in diesem Bereich.

\section{Fallbezogene Arbeit}

\subsection{Patientenvorstellung}
Der Patient, der aus Datenschutzgründen anonymisiert wird, ist ein 68-jähriger Mann,
der aufgrund einer fortgeschrittenen Lungenerkrankung beatmungspflichtig wurde.
Er leidet unter einer chronisch obstruktiven Lungenerkrankung (COPD) im Endstadium,
die eine dauerhafte maschinelle Unterstützung der Atmung erforderlich macht.

\subsection{Grunderkrankung}
COPD ist eine progressive Lungenerkrankung, die durch eine anhaltende Atemwegsverengung
gekennzeichnet ist. Sie wird durch Rauchen, Umweltfaktoren oder genetische
Dispositionen begünstigt. Typische Symptome sind Atemnot, Husten und vermehrte
Schleimbildung. Die medikamentöse Behandlung umfasst Bronchodilatatoren, inhalative
Kortikosteroide und gelegentlich Antibiotika bei Exazerbationen. In fortgeschrittenen
Fällen wird eine invasive oder nicht-invasive Beatmung notwendig.

\subsection{Nebendiagnosen}
Zusätzlich zur COPD wurden bei dem Patienten arterielle Hypertonie und eine chronische
Niereninsuffizienz festgestellt. Diese Erkrankungen beeinflussen die Beatmungstherapie,
da Flüssigkeitsretentionen zu Lungenödemen führen können. Dies erfordert eine
sorgfältige Anpassung der Beatmungsparameter und eine engmaschige Überwachung.

\subsection{Beatmungsgerät und Beatmungsform}
Der Patient wird mit einem modernen Intensivbeatmungsgerät versorgt, dem Dräger Evita
500. Die gewählte Beatmungsform ist PCV (Pressure Controlled Ventilation). Diese Form
der Beatmung wird genutzt, um die Lunge vor übermäßigen Drücken zu schützen und eine
kontrollierte Oxygenierung sicherzustellen. Der Modus wird regelmäßig an den Zustand
des Patienten angepasst.

\subsection{Beatmungsparameter}
Die für den Patienten eingestellten Werte am Beatmungsgerät sind:
\begin{itemize}
	\item PEEP: 8 cmH$_2$O
	\item FiO$_2$: 40%
	\item Atemfrequenz: 16/min
	\item Inspirationsdruck: 18 cmH$_2$O
\end{itemize}
Diese Parameter sind so gewählt, dass eine ausreichende Oxygenierung sichergestellt
und eine Hyperkapnie vermieden wird.

\subsection{Alarmgrenzen der Beatmung}
Die Alarmgrenzen dienen dazu, kritische Zustände frühzeitig zu erkennen und
entsprechend zu reagieren. Beim Patienten sind folgende Alarmgrenzen eingestellt:
\begin{itemize}
	\item Obere Druckgrenze: 30 cmH$_2$O
	\item Untere Druckgrenze: 5 cmH$_2$O
	\item Apnoe-Alarm: 20 Sekunden
\end{itemize}
Diese Werte wurden auf Basis der individuellen Lungenfunktion des Patienten festgelegt.

\subsection{Beatmungszugang}
Der Patient wird invasiv über eine Trachealkanüle (TK) beatmet. Eine nicht-invasive
Beatmung (NIV) war aufgrund der Schwere der Erkrankung nicht mehr möglich. Nachfolgend
eine Gegenüberstellung der Beatmungszugänge:

\begin{table}[h]
	\centering
	\begin{tabular}{|l|c|c|}
		\hline
		                    & NIV     & Invasive Beatmung \\
		\hline
		Mobilität           & Hoch    & Eingeschränkt     \\
		Weaning-Möglichkeit & Gut     & Gering            \\
		Infektionsrisiko    & Niedrig & Hoch              \\
		\hline
	\end{tabular}
	\caption{Vergleich der Beatmungszugänge}
\end{table}

\subsection{Pflegetherapeutische Maßnahmen}
Um die bestmögliche Versorgung des Patienten sicherzustellen, werden folgende
pflegerische Maßnahmen durchgeführt:
\begin{itemize}
	\item Inhalationstherapie zur Sekretolyse
	\item Regelmäßiges Absaugen zur Entfernung von Bronchialsekreten
	\item Lagerungswechsel zur Pneumonieprophylaxe
	\item Mobilisation im Rahmen der physiotherapeutischen Betreuung
\end{itemize}
Diese Maßnahmen tragen zur Erhaltung der Lungenfunktion und zur Prävention von
Komplikationen bei.

% Fazit
\section{Fazit}
Die Betreuung beatmungspflichtiger Patienten erfordert ein interdisziplinäres
Zusammenwirken von Pflegekräften, Ärzten und Therapeuten. Der hier dargestellte
Fall zeigt, wie wichtig eine individuelle Anpassung der Beatmungstherapie ist.
Langfristig könnten technologische Fortschritte eine bessere Autonomie der Patienten
ermöglichen und die Beatmungssituation weiter optimieren.

\newpage
\bibliographystyle{plain}
\bibliography{Facharbeit}

\end{document}
