\documentclass[a4paper,12pt]{article}
\usepackage[utf8]{inputenc}
\usepackage[T1]{fontenc}
\usepackage[ngerman]{babel}
\usepackage{graphicx}
\usepackage{geometry}
\usepackage{setspace}
\usepackage{float}
\usepackage{hyperref}
\usepackage{tabularx}
\usepackage{booktabs}
\usepackage{cite}
\usepackage{array}
\usepackage{longtable}

\geometry{a4paper, left=2cm, right=2cm, top=2cm, bottom=2cm}
\onehalfspacing

\setlength{\parindent}{0pt}

\usepackage{helvet}
\renewcommand\familydefault{\sfdefault}

\newcommand{\autor}{Elyssa Jane Dean}
\newcommand{\geburtsdatum}{27.09.1992}
\newcommand{\arbeitgeber}{IUSCare GmbH Ambulante Intensivpflege}
\newcommand{\kursbezeichnung}{FAI: Fachkraft für außerklinische Intensivpflege}
\newcommand{\kurszeitraum}{27.01.2025 - 04.02.2025}
\newcommand{\abgabedatum}{11.03.2025}

\title{\textbf{Facharbeit}}
\author{
    \autor \\
    Geburtsdatum: \geburtsdatum \\
    Arbeitgeber: \arbeitgeber \\
    Kursbezeichnung: \kursbezeichnung \\
    Zeitraum des Kurses: \kurszeitraum \\
    Abgabedatum: \abgabedatum
}
\date{}

\begin{document}

\begin{titlepage}
	\centering
	\vspace*{2cm}
	\vspace{1cm}

	{\Huge \textbf{Facharbeit} \par}
	\vspace{1.5cm}

	{\Large \textbf{\autor} \par}
	\vspace{0.5cm}

	{\large Geburtsdatum: \geburtsdatum \par}
	\vspace{0.5cm}

	{\large Arbeitgeber: \arbeitgeber \par}
	\vspace{0.5cm}

	{\large Kursbezeichnung: \kursbezeichnung \par}
	\vspace{0.5cm}

	{\large Zeitraum des Kurses: \kurszeitraum \par}
	\vspace{0.5cm}

	{\large Abgabedatum: \abgabedatum \par}
	\vspace{2cm}

	\vfill
	{\large \today \par}
\end{titlepage}

\tableofcontents
\newpage

\section{Einleitung}
Ich bin \autor. Ich habe meinen insgesamt fünfjährigen
Bachelorabschluss in Nursing auf den Philippinen im Jahr 2012 abgeschlossen
und meine Anerkennung als Gesundheits- und Krankenpflegerin (GKP) in
Deutschland im März 2024 erhalten.
Ich habe dreieinhalb Jahre in Vollzeit in einer orthopädischen Klinik in
Braunfels gearbeitet, bevor ich im Oktober 2024 zur Konstant Ambulanten
Intensivpflege gewechselt bin, die nun als \arbeitgeber bekannt ist.
Zusätzlich habe ich zwei Jahre in den Philippinen in der medizinischen und
chirurgischen Abteilung gearbeitet und anschließend drei Jahre im OP-Complex.
Diese Abteilung umfasst den Operationssaal, den Aufwachraum, die Endoskopie
und den Entbindungssaal.
Ich habe mich entschieden, in die ambulante Intensivpflege zu wechseln, da ich
das Bedürfnis hatte, eigenständiger zu arbeiten und Neues zu lernen. Mein
Wissen über das Thema „Beatmung“ war begrenzt, weshalb ich mich intensiver
damit beschäftigen wollte.

\section{Biografie und Vorstellung des Patienten}
Mein Patient ist männlich, 82 Jahre alt, wurde am 14.12.1942 geboren
und hat Pflegegrad 4. Er ist verwitwet und 
lebt derzeit mit seiner Lebensgefährtin zusammen.
In seiner Freizeit spielt er gerne Englisch Horn 
und war Mitglied einer Kurverwaltung in Bad Orb.
Im Juni 2015 wurde bei ihm ein Tonsillenkarzinom links diagnostiziert. Im
August 2015 erfolgte eine Tonsillektomie links mit anschließender primärer
Radiochemotherapie im Universitätsklinikum Gießen und Marburg (UKGM).
Am 15.06.2024 wurde eine Hypopharynx-Tamponade sowie ein plastisches
Tracheostoma mit operativer Ligatur der A. carotis externa rechts angelegt.
Vom 24.06.2024 bis zum 08.08.2024 befand sich der Patient in den
Asklepios-Schwalm-Eder-Kliniken aufgrund einer respiratorischen
Globalinsuffizienz mit Azidose und CO$_2$-Narkose infolge einer
Sekretverlegung im Tracheostoma. Dies führte zu einem beginnenden septischen
Schock. Während dieses Aufenthalts wurde er am 26.06.2024 und 08.07.2024 mit
einer BiPAP-Beatmung beatmet. Am 02.07.2024 wurde eine PEG-Sonde angelegt,
da er unter Schluckstörungen leidet.

\section{Grunderkrankung: Mandelkrebs}

\subsection{Definition}
Mandelkrebs, auch als Tonsillenkarzinom bezeichnet, ist eine bösartige
Erkrankung der Gaumenmandeln. Sie gehört zur Gruppe der Mund-Rachen-Tumoren
(Oropharynxkarzinome). Männer sind dreimal häufiger betroffen als Frauen.
Das größte Risiko für die maligne Entstehung besteht bei regelmäßigem oder
exzessivem Konsum von Tabak und Alkohol. Zudem stellt eine Infektion mit
Humanen Papillomviren einen wesentlichen Risikofaktor dar.

\subsection{Merkmale der Grunderkrankung}
Da es keine spezifischen Vorstadien gibt, ist Mandelkrebs oft schwer frühzeitig
zu erkennen. Zu den unspezifischen Symptomen gehören Schluckbeschwerden,
Schwellungen im Halsbereich, anhaltende Heiserkeit, Husten, Mundgeruch sowie
Probleme beim Essen und Trinken.

\subsection{Diagnose}
Zur Feststellung einer Diagnose können verschiedene Untersuchungen durchgeführt
werden. Dazu gehören die Spiegeluntersuchung, die Panendoskopie, die Biopsie,
die Sonographie, die Computertomographie, die Magnetresonanztomographie sowie
die Szintigraphie des Rachenraumes. Der Tumor wird anschließend klassifiziert,
unter Berücksichtigung von Metastasen in benachbarten Organen, um eine geeignete
Therapie festzulegen. Die Stadieneinteilung des Mandelkrebses unterscheidet sich
nicht wesentlich von der allgemeinen T-Klassifikation der Tumoren. Je nach Größe
oder Ausdehnung des Tumors gibt es folgende Stadien: Im T1-Stadium ist der Tumor
kleiner als 2 cm, im T2-Stadium hat der Tumor eine Größe von 2 bis 4 cm. Das T3-Stadium
beschreibt einen Tumor, der größer als 4 cm ist. Im T4-Stadium hat der Tumor
unabhängig von der Größe umliegende Gewebestrukturen infiltriert. Am häufigsten
sind hiervon der Hals, die Wangen und die Zungengrundmuskulatur betroffen.

\subsection{Medikamentöse Therapie}

Der Patient hat aktuell folgende Medikamente nach AAO vom 8.8.2024.

\begin{longtable}{|p{4cm}|p{2cm}|p{2cm}|p{3cm}|p{5cm}|}
	\hline
	\textbf{Wirkstoff}     & \textbf{Stärke} & \textbf{Form} & \textbf{Täglich} & \textbf{Hinweise}        \\
	\hline
	Insulin lispro         & 100 E           & Lösung        & nach Schema      &                          \\
	\hline
	Melperon               & 25 mg           & Tabl          & 0-0-1            &                          \\
	\hline
	Torasemid              & 10 mg           & Tabl          & 1-0-0            &                          \\
	\hline
	Salbutamol             & 5 mg            & LOV           & nach Bedarf      & zum Inhalieren           \\
	\hline
	Ipratropium Kation     & 0.25 mg         & LOV           & nach Bedarf      & zum Inhalieren           \\
	\hline
	Piritramid             & 7.5 mg/ml       & Amp           & ½ Amp. s.c.      & Bedarfsschmerzmedikation \\
	\hline
	Insulin lispro isophan & 25/75           & Susp          & 30 IE            &                          \\
	\hline
	Acetylsalicylsäure     & 100 mg          & Tabl          & 1-0-0            &                          \\
	\hline
	Atorvastatin           & 20 mg           & Tabl          & 1-0-0            &                          \\
	\hline
	Escitalopram           & 10 mg           & Tabl          & 1-0-0            &                          \\
	\hline
	Ciprofloxacin          & 500 mg          & Tabl          & 1-0-0            & bis Packungsende         \\
	\hline
	Mirtazapin             & 15 mg           & Tabl          & 0-0-1            &                          \\
	\hline
	Lorazepam              & 1 mg            & Tabl          & ½ Tbl. über PEG  & Bedarf max. 3x tägl.     \\
	\hline
\end{longtable}

\subsection{Verlauf}
In den meisten Fällen wurde der Tumor mit einem großen Sicherheitsabstand chirurgisch
entfernt. Nach der Operation folgte in einigen Fällen eine Radiotherapie, Chemotherapie
oder eine kombinierte Radiochemotherapie. Die Prognose hängt sehr stark vom Stadium
der Diagnose ab, und Metastasen beeinflussen den Krankheitsverlauf negativ. So ergibt
sich für den Mandelkrebs eine 5-Jahres-Überlebensrate, die je nach Stadium unterschiedlich
ist. Im T1-Stadium leben etwa 80–90\% der Patienten noch fünf Jahre nach der Diagnose,
im T2-Stadium sind es ca. 70–75\%, im T3-Stadium liegt die Überlebensrate bei etwa
40–50\%, und im T4-Stadium beträgt sie nur noch ca. 10–35\%.

\subsection{Nebendiagnose}

\bibliographystyle{plain}
\bibliography{Facharbeit}

\end{document}
